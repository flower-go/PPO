\chapter{Overcooked environment}

\section{Overcooked game}
Before we get into our problems with cooperation let us first examine the environment. 
We will be working with environment based on popular cooking video game \url{https://ghosttowngames.com/overcooked/}.
Overcooked is multiplayer cooperative game where the goal is to work in a kitchen as a team with partner cooks and prepare together various dishes within limited time.
However, the game is dynamic to a great extent. In many maps the kitchen itself is not static and may be changing on a run. 
Moreover, random events such as pots catching fire add to the chaos. The challenge lies in coordination with rest of the team and dividing subtasks efficiently.

\par
The aforementioned game was simplified and reimplemented to simpler environment \url{https://github.com/HumanCompatibleAI/overcooked_ai} to serve a purpose of scientific common ground for studying multi agent cooperation in somehwat complex settings.
Lot of additional features of original game were removed and remained only essential coordination aspects.
In its simplest form, environment is taking place in small static kitchen layout where only available recipe is onion soup which can be prepared by putting three onions in a pot and waiting for given time period.
Somewhere in the kitchen there is unlimited source of onions and dish dispenser, where player can grab a dish to carry cooked onion soup in to the counter.
Team of cooks is rewarded as team by abstract reward of value 20 every time cooked soup is delivered to the counter. 
It may seem that the task is quite straightforward. However, players face problems on multiple levels.

\section{Basic layouts}
Although the Overcooked implementation has its own generator that can be used to generate new random kitchen layouts, the majority of the related scientific work has so far experimented with a fixed set of predefined layouts, where each of them capture some important aspect of coordination.

\par 
\subsection*{Cramped room}
\begin{center}
    \includegraphics*[width=4cm]{../img/cramped_room_layout.png}
\end{center}
Cramped room as a name suggests represents cramped kitchen layout where all important places are relatively easy to reach. Challenge lies in low level coordination of movement with the other partner as there is no spare room.

    

\subsection*{Assymetric advantages}
\begin{center}
    \includegraphics*[width=6cm]{../img/asymmetric_advantages_layout.png}
\end{center}
In Assymetric advantages both players are located in separated regions where each region is fully self-sustaining. However, each region has better potential for specific subtask. 
And it is only when both players make the most of their own region's potential that the maximal shared efficacy is reached.

\subsection*{Coordination ring}
\begin{center}
    \includegraphics*[width=4cm]{../img/coordination_ring_layout.png}
\end{center}
The Coordination ring is another example of a layout where clever coordination is required as the only possible movement around the kitchen is along a narrow circular path that can be used in a given direction.
For example, if one player decides to move in clockwise direction, the other player would automatically get stuck if persuing counter-clockwise movement.

\subsection*{Forced coordination}
\begin{center}
    \includegraphics*[width=4cm]{../img/forced_coordination_layout.png}
\end{center}
Forced coordination kitchen layout is significantly different from others. 
In this layout, each player is located in a separate region where neither player has all the resources necessary to prepare a complete onion soup. 
Thus, players are forced to cooperate with each other with the resources they have.

\subsection*{Counter circuit}
\begin{center}
    \includegraphics*[width=6cm]{../img/counter_circuit_layout.png}
\end{center}
In the last layout, the situation may look similar to the coordination ring. However, in this case, carrying onions around the entire kitchen is highly suboptimal no matter which direction the players choose. 
To deliver onions efficiently, players must pass them over the counter to shorten the distance. 
However, the cooks still need to decide who will be responsible for bringing the plates. 


\section{Environment description}
\subsection{Actions space, episode horizont, shaped rewards, state representation MLP vs CNN}
\subsection{Reset state static position, index switching, Randomization function}







